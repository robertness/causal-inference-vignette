\documentclass{letter}[11]

\usepackage{fullpage}
\usepackage{setspace} 
\usepackage{algorithm}
\usepackage{algorithmic}
\usepackage{amsmath}

\usepackage{graphicx}
%\usepackage{subfig} 
\usepackage{multirow} %for multiple rows in table
\usepackage{rotating}  % for vertical text
\usepackage{array}  % for vertical alignment in a cell

\usepackage{hyperref}
\usepackage{verbatim} %comment over multiple lines

\usepackage[table]{xcolor}
\usepackage{multirow} %for multiple rows in table
\usepackage{rotating}  % for vertical text
\usepackage{color}

\usepackage{pdfpages}

\def\todo#1{{\color{red}[For Robert to finish: #1]}}
\def\ov#1{{\color{magenta}#1}}

\def\r#1{{\begin{quote}\textsf{\color{blue} #1}\end{quote}}}
\begin{document}
{
\address{
{\rm Olga Vitek} \\
{\rm Sy and Laurie Sternberg Associate Professor} \\
{\rm College of Science} \\
{\rm College of Computer and Information Science} \\
{\rm Northeastern University} \\
{\tt o.vitek@neu.edu}}
\signature{Olga Vitek}

\begin{letter}{
 		{\em Journal of Proteome Research}\\
		Attn: Susan Weintraub \\
		Editorial Office
}

\opening{Dear Dr. Weintraub:}   
\thispagestyle{empty}

We received the reviewers' comments on our manuscript {\it From correlation to causality: statistical approaches to learning regulatory relationships in large-scale biomolecular investigations}, and would like to thank them for the positive feedback.

On the following pages of this letter we discuss each reviewer's concerns (\textsf{\color{blue} reproduced in blue in sans serif font}), and the associated response. We hope that the reviewers will find the responses adequate, and look forward to hearing from you soon.\\ \\

\begin{tabular}{cc}
~~~~~~~~~~~~~~~~~~~~~~~~~~~~~~~~~~~~~~~~~~~~~~~~~~~~~~~~~~~~~~~~ & Sincerely,\\
\hspace{10cm} &  \\
& Olga Vitek, PhD 
\end{tabular}
\vfill 



\end{letter}}
%%%%%%%%%%%%%%%%%%%%%%%%%%%%%%%%%%%%%%%%%%%%%%%%%%%%%%%%%%%%%%%%%
\newpage
\normalcolor
\noindent {\Large \bf Reviewer 1} 
\vspace{3mm} \\

\r{This paper discusses some of the difficulties of inferring networks and pathways with high-throughput data.  It is pedagogical --- methods and their short-comings are illustrated through simulation results. No new methods are proposed. The paper is well-written and should be useful for the proteomics community.}

We thank the reviewer for the positive feedback.

\r{The authors suggest single cell analysis as a means of increasing sample size. Single cell ``omics" in which all the cells come from a single biological replicate are useful for learning pathways because the cells are generally in different states, and thus provide information about dependencies within the individual. However, any study in which a single individual is observed, provides information only about that individual --- inference to a broader population must then be made either on purely theoretical grounds, or based on follow-up studies which confirm that the same relationship exists in others.}

We wholeheartedly agree with the reviewer regarding the importance of true biological replication. Indeed, this is currently insufficiently emphasized in the network inference literature. The revised manuscript now mentions \todo{in section XXX, page XXX ... rewrite in in terms of the true bioogical replication, and the fact that single sells are (heterogeneous) instances of subsampling}of the fact that while single cell data provides more data inputs to computational method, it is not sufficient for population level inference if the cells were not acquired from a sufficiently large sample of subjects from that population.

\r{p.11 line 35 ``depending on whether the analytes share a same or temporal context.'' There appears to be some missing text.}

Changed to ``depending on whether the analyses share the same spatial or temporal context''.

\r{p.12 line 38: ``my'' seems out of place}

Corrected.


%%%%%%%%%%%%%%%%%%%%%%%%%%%%%%%%%%%%%%%%%%%%%%%%%%%%%%%%%%%%%%%%%%%%%%%%%%%%%%%%%%%%%%%
\newpage
\normalcolor
\noindent {\Large \bf Reviewer 2} 
\vspace{3mm} \\

\r{p.1 line 7. not sure what are ``statistical associations" and that ``causal inference" deals with statistics only.}

We clarified \todo{In section XXX}  ``statistical associations'' with examples of statistical measures of association familiar to biologists, and that while ''causal inference'' spans multiple domains, we only address the specific case of causal inference through the statistical design of experiments.

\r{p.1 line 15 ``large amount of data generated by these experiments make the task of causal inference more difficult".  It would be great either to refer a publication/story where it was demonstrated, or to explain why authors think so and what they put into the word ``difficult" .}

\todo{The reviewer refers to the shortcoming of our paper, which we previously discussed, that the paper does not cite examples of network inference from high-throughput proteomic/mass spectrometry (or other types of) data. Here I would startb by adding this brief literature review, and then comment using these specific examples on the nature of difficulty. Some were successful by chance. Others had to simplify the problem. Explain why there are not many examples. Be specific when describing the difficulty. }

One of the goals of this manuscript is to demonstrate this difficulty exists and therefore argue that due to this difficulty high content low sampling throughput experiments are unsuitable for causal inference.  We agree that as stated, it seems as though we take this difficulty for granted.  We clarified the language and also added references to key concepts in statistical causal inference when they are introduced later in the manuscript.

\r{p.1 line 17 ``associations arising purely from random chance". On the other hand, this can happen (with a small probability) in an individual experiment as well, if the associations are random.}

We agree, and we modified the language to show that while even in the simplest single experiment with a single treatment and univariate response, a strong statistical association between treatment and response could have arisen spuriously.  We tied this in with the p-value, a statistical concept that addresses the problem of spurious results and which is familiar to biological investigators.

\todo{Not sure we need p-values here. I would rather say that the probability of random association is much higher in large-scale experiments}

\r{p.4 figure C. Using more different symbols in C (rather than solid triangles / circles) would certainly improve readability of the figure.}

We changed the triangles to squares, to increase readability.

\r{p.4 figure C. It seems like there is a circle with Raf=0.7, Erk=0.5 missing in C, left plot.}

The bounds on the plots were corrected.

\r{p.4 line 52. Why does the lack of correlation for samples with high Mek indicate conditional independence?}

We agree that the link between low correlation and conditional independence needs to be made more explicit.  To this end, we added to this example a statistical test for conditional independence between Raf and Erk given Mek.  The null hypothesis is conditional independence between Raf and Erk, and thus conditional correlation between Raf and Erk is a test statistic.  So when conditional correlation is lower than some statistically significant threshold, the null hypothesis cannot be rejected.  This ties the  intuition of the plots to a familiar statistical testing workflow for making decisions on the presence of conditional dependence given data.  

\todo{It's a different problem. The reviewer did not understand that while the figure only shows a subset of the data as an illustration, the actual calculation involves full conditional probability distributions. Do not add any tests, but better explain the meaning of conditional independence, and how it is used.}

\r{p.4 line 52. Can it be the case that the scatterplots B and C look as in this example, but there is a direct dependence from Raf to Erk?  p.5 line 18. ``If the values of concentrations of two proteins vary between the biological samples in a coordinated manner, due to a common biological mechanism, the proteins are called statistically dependent."  Am I right that this def of the stat dependence objects your example of independence between Ref and Erk? Also, not clear to me: the mechanism of dependence between Ref and Erk is pretty ``biological" and even known (through Mek). So why can't we call Ref and Erk statistically dependent?} 

We clarified the language to highlight the fact that Raf and Erk are simultaneously dependent and conditionally independent.  We also clarified the concept of a ``common biological mechanism" to mean that two proteins are affected by a common regulatory cause.  Finally, we made sure that statistical association and probabilistic dependence are clearly distinguished, and eliminated all references to ``statistical dependence" --- a potential source of confusion.

\todo{Right. The reviewer is not familiar with conditional independence and simply needs more non-technical explanation}
\todo{Spell out the definitions here - the reviewer may not read the revised manuscript, but will read the letter}

\r{p.5 line 30.  What are indirect statistical associations.}

We clarified this language to ``statistical associations resulting from indirect regulatory relationships".

\todo{Spell out the definitions here - the reviewer may not read the revised manuscript, but will read the letter}


\r{p.5 line 42 re: ``direct causal events can be distinguished from other undesirable types of associations by controlling for the intermediaries and the common regulators":  The interesting Q: when they can be distingushed? E.g. if the function from Erk is identical to Mek then it's not possible to distinguished between association Ref-Mek and Ref-Erk.}

This is true, and to address this we briefly previous work by co-author Karen Sachs with gives protein signaling examples of the conditions required to disentangle conditional dependence.

\r{p.5 line 50 ``The absence of statistical association between two proteins, when the intermediates or the common regulators are controlled at a fixed level, is called conditional independence." Hmm.}

Indeed this statement, as it was phrased, is not accurate.  Independence and conditional independence are not defined in terms of statistical associations in the data, and as mentioned above, strong statistical associations can arise even when there is independence/conditional independence.  We corrected the language here emphasized this point here and elsewhere else in the manuscript.

\todo{Do not say many words, but instead provide the correct definition right here}

\r{p.6 18 ``This phenomenon --- the disappearance of the association between Raf and Erk upon conditioning on Mek --- is evidence that Raf and Erk are conditionally independent."  Why is it evidence?}

We use the statistical test for conditional independence, described above in our response to the comment on p.4 line 52, to describe how to make decisions on the presence of conditional dependence  given data.

\todo{Do you rely on this test for network inference, or do you only do this because of the reviewer? If it's for the reviewer then do not worry about that. Just explain again in layman terms.}

\r{p.8 Figure 3.  Hm, not very convincing since the values are still pretty low, lower than 0.8 illustrated in the figure 2. Maybe increase the number of proteins to 1,000,000?}

We ran this simulation at 1 million and generated a new plot.  This resulted in higher spurious correlation values.

\r{p.9 figure 4. Hm, maybe the Margaritis algorithm is bad?}

Indeed, we agree that it is important to point out the various algorithms for conducting tests of conditional independence have different approaches, and therefore could have different results.  We regenerated the graph and overlaid the results of the Margaritis algorithm with 2 competing algorithms. 

\r{p.9 line 48. At this point, it'd be great to have a figure similar to Figure 3 but that would illustrate that increasing the number of replicates helps to reduce the probability of finding a spurious dependence. p.10 line 38: ``Limit the number of analytes." Sure enough, nobody wants reducing the coverage. Can you please convince readers by providing a sort of rule of thumb? For example, ``Considering half the analytes increases the confidence in 2 times."}

To answer these points we conducted a simulation were we varied the sampling and content throughput of the input dataset, and evaluated detection of conditional dependence relationships while controlling for type 1 error.  We then generated a plot where the X axis is the ratio of sample size to coverage size, and the Y access is true positive rate.  We overlaid two curves, one where the Gaussian assumption was applied, and one where the data is assumed nonlinear.  We then highlighted ranges on the X axis that could be covered by common proteomics technologies.  This will give investigators a clear guide of what they can expect from a given technology when doing such an experiment.

\r{p.9 line 48.  What would be great as well to have a hint what to do with the replicates? Ok, I put efforts collecting data for replicates. What's next to get my results more credible? Experimental design? Which methods?  p.10 line 53. Again, it's probably not just replicates, but also the stat methods that use them in a clever way. Which methods?}

We included a remark \todo{In sec XXX} that single cells, while providing more experimental units, do not substitute for having multiple biological replicates in order to make population inferences. \todo{Again speak in terms of biological replication and subsampling}

\r{p.10 line 9 ``Indeed, a set of interventions on < k variables may be sufficient to infer causal events between k analytes" Sounds like you have a reference for this for this fact :)}

We included references on optimal experimental design for causal inference.                                                                                                                                          



\r{p.11 line 38 ``4: Select targeted interventions wisely." p.12 line 6 ``5: Consider more broad scale interventions.": Do I understand correctly that 4 contradicts 5?}

\todo{Clarify that broad interventions mean single perturbations that simultaneously affect multiple nodes}
We clarified the language here, and referred to the specific classes of perturbagens that are available to investigators of cell signaling eg.  stimuli such as ligands of cytokines, genetic knockdowns, and siRNA or small molecule inhibitors.

%%%%%%%%%%%%%%%%%%%%%%%%%%%%%%%%%%%%%%%%%%%%%%%%%%%%%%%%%%%%%%%%%
\end{document}

